\documentclass{article}
\usepackage{amsmath, amssymb, amsthm}
\usepackage{hyperref}

\begin{document}

\section*{Product Spaces}

Suppose \(X_1, \ldots, X_n\) are arbitrary topological spaces. On their Cartesian product \(X_1 \times \cdots \times X_n\), we define the \emph{product topology} to be the topology generated by the following basis:
\[
\mathcal{B} = \{U_1 \times \cdots \times U_n : U_i \text{ is an open subset of } X_i, i = 1, \ldots, n\}.
\]

\textbf{Exercise 3.25.} Prove that \(\mathcal{B}\) is a basis for a topology.

The space \(X_1 \times \cdots \times X_n\) endowed with the product topology is called a \emph{product space}. The basis subsets of the form \(U_1 \times \cdots \times U_n\) are called \emph{product open subsets}.

For example, in the plane \(\mathbb{R}^2 = \mathbb{R} \times \mathbb{R}\), the product topology is generated by sets of the form \(I \times J\), where \(I\) and \(J\) are open subsets of \(\mathbb{R}\). A typical such set is an open rectangle.

\textbf{Exercise 3.26.} Show that the product topology on \(\mathbb{R}^n = \mathbb{R} \times \cdots \times \mathbb{R}\) is the same as the metric topology induced by the Euclidean distance function.

The product topology has its own characteristic property. It relates continuity of a map into a product space to continuity of its component functions. In the special case of a map from \(\mathbb{R}^m\) to \(\mathbb{R}^n\), this reduces to a familiar result from advanced calculus.

\subsection*{Theorem 3.27 (Characteristic Property of the Product Topology)}
Suppose \(X_1 \times \cdots \times X_n\) is a product space. For any topological space \(Y\), a map \(f : Y \to X_1 \times \cdots \times X_n\) is continuous if and only if each of its component functions \(f_i = \pi_i \circ f\) is continuous, where \(\pi_i : X_1 \times \cdots \times X_n \to X_i\) is the canonical projection:
\[
\begin{array}{ccc}
 & X_1 \times \cdots \times X_n & \\
 & \uparrow \pi_i & \\
Y & \xrightarrow{f} & X_i \\
 & \searrow f_i & \\
\end{array}
\]

\noindent
\textbf{Proof.} Suppose each \(f_i\) is continuous. To prove that \(f\) is continuous, it suffices to show that the preimage of each basis subset \(U_1 \times \cdots \times U_k\) is open. A point \(y \in Y\) is in \(f^{-1}(U_1 \times \cdots \times U_k)\) if and only if \(f_i(y) \in U_i\) for each \(i\), so
\[
f^{-1}(U_1 \times \cdots \times U_k) = f_1^{-1}(U_1) \cap \cdots \cap f_n^{-1}(U_n).
\]
Each of the sets in this intersection is open in \(Y\) by hypothesis, so it follows that \(f\) is continuous.

\end{document}